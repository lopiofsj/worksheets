%
% Word problems answers.
% For use with Calif. Pre Algebra circa 2011
% Copyright Lopiofsj
%
\documentclass[12pt]{article}
\pdfoutput=1
\begin{document}

\large

\textbf{ Answers }

\begin{enumerate}
\item Engineers design a roller coaster with a top speed of 25MPH (Miles Per Hour).  Find the top speed in FPS (Feet Per Second).

	\( \frac{110}{3} = 36\frac{2}{3} = 36.\overline{6} \) FPS

\item Matthew makes coffee during the week using 8 Tablespoons (T) for 6 Cups (C).  During the weekend he uses 10 T for 8 Cups.
	\begin{enumerate}
		\item Find the unit rate for the weekday coffee.
		\item Find the unit rate for the weekend coffee.
		\item Determine which is the strongest cup of coffee.
		\item How much coffee should be used on the weekend to match the weekday coffee?
	\end{enumerate}

	Weekday: \( 1.\overline{3} = 1\frac{1}{3} \) TPC

	Weekend: 1.25 TPC

	Stronger coffee during weekday

	Amount of coffee: \( 10.\overline{6} = 10\frac{2}{3} \) Cups

\item There are 24 banana nut muffins and 9 blueberry muffins.  What is the ratio of blueberry muffins to banana nut muffins?

	\( 3:8 = \frac{3}{8} \)

\item The ratio of Kisses to Chocolate Santas is 5:2.  If there are 15 Kisses, how many Chocolate Santas are there?

	6

\item Simon wishes to travel from her home in San Jose to 6 different cities.  5 of the cities are in different countries that use each use a different currency.  If Simon has saved enough money to spend \$1000.00 in each city, how much in each city will she have in the local currency?

	\begin{tabular}{ | c | c | }
	\hline
	\bf{City} & \bf{Currency} \\ \hline \hline
	New York & Dollar \\ \hline
	London & Pound \\ \hline
	Paris & Euro \\ \hline
	St. Petersburg & Ruble \\ \hline
	Beijing & Yuan \\ \hline
	Tokyo & Yen \\ \hline
	\hline
	\end{tabular}

	\begin{tabular}{ | c | c | }
	\hline
	\multicolumn{2}{|c|}{\bf{Currency Conversion}} \\ \hline \hline
	2 Pound & 1.5 Dollar \\ \hline
	2 Euro & 1 Pound \\ \hline
	1.5 Ruble & 3 Euro \\ \hline
	1.25 Yuan & 1.25 Ruble \\ \hline
	1 Yen & 2 Yuan \\ \hline
	5 Dollar & 1 Yen \\ \hline
	\hline
	\end{tabular}

	\begin{tabular}{ | c | c | c | c | }
	\hline
	\bf{City} & \bf{Currency} & \bf{Conversion to Dollar} & \bf{Amount} \\ \hline \hline
	New York & Dollar & 1:1 & 1000 \\ \hline
	London & Pound & 1:3 & \( 333.\overline{3} \) \\ \hline
	Paris & Euro & 2:3 & \( 666.\overline{6} \) \\ \hline
	St. Petersburg & Ruble & 1:3 & \( 333.\overline{3} \) \\ \hline
	Beijing & Yuan & 2:5 & 400 \\ \hline
	Tokyo & Yen & 1:5 & 200 \\ \hline
	\hline
	\end{tabular}
%\item \( \)
\end{enumerate}

\end{document}

