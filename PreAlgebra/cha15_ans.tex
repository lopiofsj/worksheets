%
% Worksheet answers.
% For use with Calif. Pre Algebra circa 2011
% Copyright Lopiofsj
%
\documentclass[12pt]{article}
\pdfoutput=1
\usepackage{multicol}
\begin{document}
\Large

\textbf{ Chapter 1-5 worksheet }

\textbf{ Answers }

\begin{multicols}{2}
\begin{enumerate}

	\item \( \frac{3}{5} = \frac{m}{20} \) \\
		\( m = 12 \)

	\item \( 3^{2} + 4^{2} = c^{2} \) \\
		\( c = 5 \)

	\item \( 21 - \frac{x^{2}}{7} = 14 \) \\
		\( x = 7 \)

	\item \( \sqrt{12^{2}} \) \\
		12

	\item \( \sqrt{z^{4}} \) \\
		\( z^{2} \)

	\item \( (3.6 \times 10^{-2})(2 \times 10^{3}) \) \\
		72

	\item \( w \cdot w^{-1} \) \\
		1

	\item \( \frac{4}{5}x + 3 = 7 \) \\
		\( x = 5 \)

	\item \( \frac{z - 3}{2} = 6 \) \\
		\( z = 15 \)

	\item \( \sqrt{25x^{2}} + \frac{1}{2} = 30 \) \\
		\( x = \frac{59}{10} = 5.9 \)

	\item \( (x^{2})^{3} \) \\
		\( x^{6} \)

	\item \( x \cdot x \) \\
		\( x^{2} \)

	\item \( \sqrt{ x \cdot x } \) \\
		\( x \)

	\item \( \frac{r}{4} - 3 = -8 \) \\
		\( r = -20 \)

	\item \( -6 \cdot -2 \) \\
		12

	\item \( (-\frac{1}{3})(27) \) \\
		-9

	\item \( (-4) (8) \) \\
		-32

	\item \( 1 - 3(x+2) \) \\
		\( -3x - 5 \) or \( -3x + (-5) \)

	\item \( 10 - 2(x+5) \) \\
		\( -2x \)

	\item \( (-12x^{2})(2)(-x^{3}) \) \\
		\( 24x^{5} \)

	\item \( -(5^{2}) \) \\
		-25

	\item \( \frac{1}{7} \) \\
		\( 7^{-1} \)

	\item \( \frac{1}{x^{-1}} \) \\
		\( x \)

	\item \( 3x + 2 = -x + 18 \) \\
		\( x = 4 \)

	\item \( 8^{-1} \) \\
		\( \frac{1}{8} \)

	\item \( (\frac{1}{4})^{-2} \) \\
		16

	\item \( (\frac{1}{10})^{2} \) \\
		\( \frac{1}{100} = 0.01 \)

	\item \( (-4)^{2} \) \\
		16

	\item \( (3^{2})^{-1} \) \\
		\( \frac{1}{9} \)

	\item \( (x^{2})^{10} \) \\
		\( x^{20} \)

	\item \( ((\frac{1}{2})^{2})^{-1} \) \\
		4

	\item \( 2 \cdot 2^{-1} \) \\
		1

	\item \( (x^{2})(x^{-3}) \) \\
		\( x^{-1} = \frac{1}{x} \)

	\item \( (-2)^{2} \cdot 3 \) \\
		12

	\item \( \sqrt{64x^{40}y^{2}} \) \\
		\( 8x^{20}y^{1} = 8x^{20}y \)

	\item \( (-5z^{-1}w^{3})(-10w^{3}z^{2}) \) \\
		\( 50zw^{6} \)

	\item \( 2 \times 10^{-2} \cdot 1.2 \times 10^{3}\) \\
		\( 24 = 2.4 \times 10^{1} \)

	\item \( \sqrt{ (-5wz)(-20w^{3}z^{-1}) } \) \\
		\( 10w^{2} \)

	\item \( \frac{k}{10} + 2 = 5 \) \\
		\( k = 30 \)

	\item \( \frac{z^{2}}{20} - 17 = -12 \) \\
		\( z = 10 \)

	\item \( -\frac{3}{5} + \frac{1}{2} \) \\
		\( -\frac{1}{10} = -0.1 \)

	\item \( \frac{3}{4} + 0.25 \) \\
		1

	\item \( \frac{3c}{2} + 3 = 6c - 6 \) \\
		\( c = 2 \)

	\item \( (5 \times 10^{-1})(5 \times 10^{2}) \) \\
		\( 250 = 2.5 \times 10^{2} \)

	\item \( \sqrt{3\cdot3^{3}} \) \\
		\( 3^{2} = 9 \)

	\item \( \frac{r}{2} - 6 = 8 \) \\
		\( r = 28 \)

	\item \( 2 - 5(y + 12) \) \\
		\( -5y -58 \)

	\item \( (-3)(2x^{2})(-x^{4}) \) \\
		\( 6x^{6} \)

	\item \( \sqrt{(2xy)(8x^{-1}y^{3})} \) \\
		\( 4y^{2} \)

	\item \( \frac{7}{2} - \frac{3}{2} \) \\
		2

	\item \( \frac{1}{5} + \frac{2}{3} \) \\
		\( \frac{13}{15} \)

	\item \( -\frac{1}{7} - \frac{2}{3} \) \\
		\( -\frac{17}{21} \)

	\item \( \frac{3c}{2} - 3 = -6c + 6 \) \\
		\( c = \frac{6}{5} = 1.2 \)

%	\item \( \)
\end{enumerate}
\end{multicols}

\pagebreak

\large

\begin{enumerate}

\item If a room is 200 \(ft^{2}\) (square feet), how many square inches is the room?

	\( 28800 in^{2} = 2.88 \times 10^{4} in^{2} \)

\item Matthew and Alex have built a model train set on a 4 foot by 8 foot sheet of plywood.  If the scale of the model train set is 1:160, how many square miles is represented by the model?

	\( \frac{2}{9801} = 0.0002 = 2 \times 10^{-4} Mile^{2} \)

\item Alex was riding his bike and recorded his time and speed using his phone.  After twelve minutes the phone reported that he traveled 10 miles.
	\begin{enumerate}
		\item What is his speed in Miles Per Hour?
		\item What is his speed in Feet Per Second?
		\item Is this a resonable answer?
	\end{enumerate}

	\begin{enumerate}
		\item 50 MPH
		\item \( \frac{220}{3} = 73\frac{1}{3} = 73.\overline{3} \) FPS
		\item No.  Alex has not yet achived this speed.
	\end{enumerate}

\item Engineers design a roller coaster with a top speed of 25MPH (Miles Per Hour).  Find the top speed in FPS (Feet Per Second).

	\( \frac{110}{3} = 36\frac{2}{3} = 36.\overline{6} \) FPS

\item Matthew makes coffee during the week using 8 Tablespoons (T) for 6 Cups (C).  During the weekend he uses 10 T for 8 Cups.
	\begin{enumerate}
		\item Find the unit rate for the weekday coffee.
		\item Find the unit rate for the weekend coffee.
		\item Determine which is the strongest cup of coffee.
		\item How much coffee should be used on the weekend to match the weekday coffee?
	\end{enumerate}

	\begin{enumerate}
		\item Weekday: \( 1.\overline{3} = 1\frac{1}{3} \) TPC
		\item Weekend: 1.25 TPC
		\item Stronger coffee during weekday
		\item Amount of coffee: \( 10.\overline{6} = 10\frac{2}{3} \) Cups
	\end{enumerate}

\item There are 24 banana nut muffins and 9 blueberry muffins.  What is the ratio of blueberry muffins to banana nut muffins?

	\( 3:8 = \frac{3}{8} \)

\item The ratio of Kisses to Chocolate Santas is 5:2.  If there are 15 Kisses, how many Chocolate Santas are there?

	6

\item Simon wishes to travel from her home in San Jose to 6 different cities.  5 of the cities are in different countries that use each use a different currency.  If Simon has saved enough money to spend \$1000.00 in each city, how much in each city will she have in the local currency?

	\begin{tabular}{ | c | c | }
	\hline
	\bf{City} & \bf{Currency} \\ \hline \hline
	New York & Dollar \\ \hline
	London & Pound \\ \hline
	Paris & Euro \\ \hline
	St. Petersburg & Ruble \\ \hline
	Beijing & Yuan \\ \hline
	Tokyo & Yen \\ \hline
	\hline
	\end{tabular}

	\begin{tabular}{ | c | c | }
	\hline
	\multicolumn{2}{|c|}{\bf{Currency Conversion}} \\ \hline \hline
	2 Pound & \( \frac{2}{3} \) Dollar \\ \hline
	2 Euro & 1 Pound \\ \hline
	1.5 Ruble & 3 Euro \\ \hline
	1.25 Yuan & 1.25 Ruble \\ \hline
	1 Yen & 2 Yuan \\ \hline
	5 Dollar & 1 Yen \\ \hline
	\hline
	\end{tabular}

	\begin{tabular}{ | c | c | c | c | }
	\hline
	\bf{City} & \bf{Currency} & \bf{Conversion to Dollar} & \bf{Amount} \\ \hline \hline
	New York & Dollar & 1:1 & 1000 \\ \hline
	London & Pound & 1:3 & \( 333.\overline{3} \) \\ \hline
	Paris & Euro & 2:3 & \( 666.\overline{6} \) \\ \hline
	St. Petersburg & Ruble & 1:3 & \( 333.\overline{3} \) \\ \hline
	Beijing & Yuan & 2:5 & 400 \\ \hline
	Tokyo & Yen & 1:5 & 200 \\ \hline
	\hline
	\end{tabular}
%\item \( \)
\end{enumerate}


\copyright Matthew Lopez 2012
\end{document}
